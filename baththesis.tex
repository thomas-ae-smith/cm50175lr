\documentclass[11pt]{report}
% packages
\usepackage{baththesis}
\usepackage{amssymb} %for Blackboard bold etc
\usepackage{graphicx} %for including eps graphics
% front matter
\title{Structure-Aware Procedural Destruction} \author{Thomas AE. Smith}
\degree{Doctor of Engineering}
\department{The Centre for Digital Entertainment} \degreemonthyear{May 2014}
\norestrictions
\begin{document}

\maketitle

\begin{abstract} %32
	In this proposal and literature review a number of possible approaches to developing a procedural destruction system are investigated, and an approach developed to guide practical research and development in this area.
\end{abstract}


% 1.1 A background to program synthesis 
% 1 1.1.2 Problems with program synthesis 4
% 1.2 An alternative approach through reuse 5
% 1.2.1 Code level reuse 6
% 1.2.2 Design level reuse 6
% 1.2.3 Specification level reuse 7
% 1.2.4 The object-oriented approach 7
% 1.2.5 Abstraction for reuse 8
% 1.3 Reuse or synthesis 9
% 1.4 Combining reuse with artificial intelligence 9
% 1.5 Overview of thesis 9
\chapter{Introduction}
	% Description of a problem. All research should be problem driven. You present a problem, what others have done to solve the problem (brief summary), your solution, why it is different and what the benefits are. A relatively brief version of this comes into the introduction. Explain your motivation, why you chose this project. At the end write a clear roadmap for the rest of the document.
	% The exact structure of the document depends on the concrete type of project, but it must include a 3-5 page Introduction section which:
% • Summarises the expected main argument of the dissertation,
% • Presents a clear roadmap for the rest of the document, and
	Paragraph on video games introducing context and the drive for increased fidelity, enabled by ever-improving hardware. Also mention variety.

	Providing high-fidelity destruction requires investment. Some approaches to rendering damage applied to in-game assets scale better than others. There are a number of different axes on which investment may scale - by the level of fidelity provided by the system, or by the number of assets which may be damaged. In an environment where a variety of assets may have a wide variety of different types of damage inflicted, attempting to realistically render the results of these damage events results in a combinatorial explosion that quickly becomes unmanageable both in terms of processing it in real time and when attempting to supply the assets that support rendering damaged structures.

	A number of approaches have been take in order to attempt to automate the process both of determining what sort of damage should be rendered, and also procedurally generating the assets (meshes, textures, VFX) necessary in order to faithfully render this damage. To date, the majority of such procedural systems focus on a per-material or per-object approach that responds to a single kind of impact damage and merely varies the magnitude of the effect applied in order to respond to differences in damage kind --- i.e. the difference between a single stray bullet impact and a nearby high-yield explosive detonation.

	Where larger scale destruction effects are necessary, these are typically handled by a combination of small-scale procedurally destructible elements, and custom designer-written scripts that make use of higher level knowledge about the physical structure of the object being destroyed and the `intent' of the destruction - whether it is simply intended to provide a visual effect, or will have some ludic ramifications on the players' abilities or accessible areas within the game.

	Some genres of game --- notably space- or naval-combat sims, feature large physical structures that can be subject to [idiosyncratic] recognisable forms of destruction --- for example, hull deformation as a result of impact, or separation of smaller components such as conning towers. The types of damage that each portion of such a ship may be subject to are dependant not only on the kind of damage applied, but also on the physical structure of the area affected and the supporting physical structures surrounding it. No existing procedural destruction systems apply knowledge about the structure of the affected object when generating possible damage outcomes, and so in this document a review of related research is presented, and a promising avenue for further investigation is suggested.

	\section{Implementation Constraints}
		Loose coupling
		Processor time --- must run in real-time in a fraction of a frame, or calculations may be amortised across mutlitple frames. It is improtatnt to ensutre that the system ramains responsive, whilst also not interfereing with the mutitude of other processing requirements imposed by the game.
		As the renderings are likely to provide tactically-useful information to players during runtime, it is important to ensure that the appearance of damage is consistent across each clients' rendering. This is particularly challenging given the constrained network resources --- it will be essential to minimise the amount of additional data sent over the network merely to ensure client synchronisation.

	\section{Document Overview}

\chapter{Literature Review}
	% Now you can go into more detail on what others have done. For each contribution explain if the work is relevant to your solution, you will use the method of ‘X’ for part of the project, or if you have a way of improving on ‘X’ explain that briefly too ( a longer explanation will come later when you have done the project).
	In order to inform the defevelopment of the project, it is worth considering reserach in related areas. First, a rance of exisitng appropached to procediural destruction systems withing commercial games are considered --- accompanied by academic overviews of particular techniques wehre available. Then, the concept of declarative solvers is introduced, with particular reference to existing projects and aproaches that might provide promising initial directions. Finally, there is a brief overview of some of the other techniques that may be necessary in order to bind the abstract output of a solver system to a concreted visual representation within a game.

	\section{Existing Implementations}
		Any particular procedural destruction system is distinguished by the implementation chosen for two primary considerations: the [decision-making] approach, and the rendering solution. The [something] is often defined by the ludic environment of the system --- design decisions will be made at the gameplay level about the particular model of damage most suited to the desired in-game experience. A suitable way of communicating this information to a player via visual (rarely, audible) representation can then be developed.

		In order to provide context for the description of the proposed system, it is instructive to investigate the range of methods that have previously been used to model and represent damage and destruction within games to date. A number of the approaches described below are mutully incompatible, however there are a small number of subsets that are often used in combination with each other, and which may indicate useful techniques that could be applicable in the present domain. The sections below are arranged roughly in order of increasing implementation complexity, and therefore also chronologically in order of representation fidelity.

		\subsection{No Destruction}
			In early and/or simple games, it is commonly the case that all game objects are simply indestructible, which requires no in-game damage metric or alternative visual rendering.

		\subsection{Scripted Destruction}
			The simplest possible form or destruction is simple presence/absence of an object in response to a particular in-game event --- typically collision with a projectile fired by the player. More advanced implementations may replace the object with a visual effect such as an explosion.

		\subsection{Triggered Destruction}
			The overall `health' or `structural integrity' of an 
		in-game metric, no visual representation. Possible death animation

		\subsection{Art swap} - various thresholds, identical each time. Fidelity scales with artist investment
		\subsection{Voxel-based} approaches
		\subsection{Destructible materials}
		\subsection{Scripted systems}

		Previous research on destruction of comparatively small objects \cite{van2011procedural}.
	\section{Declarative Solvers}
		LOG-Ideah \cite{novelli2012log}
		Truth maintenance systems

	\section{Relevant Rendering Techniques}
		Decal application
		Procedural placement of VFX
		Mesh deformation
		Mesh subdivision

		One possible approach to redering deforemd structures is described by Morris et. al. in \cite{morris2012modular}, however sinve all output is determined at runtime via GPU shaders, it would be difficult to ensure cross-client synchronisation of damage appearance

\chapter{Proposed Approach}
	% You will not at this stage have the details but give as many ideas as you can on what will be tried. Methodology: depends on what subject (need to discuss with supervisor), proposed data structures and algorithms should be mentioned here.
	Paragraph on overview of approach, to provide context for
	\section{Research goals}
	As an essential part of the [] it will be necessary to develop and implement a working procedural destruction system, in order to demonstrate the feasibility of the following claims.
	\begin{itemize}
		\item Loose Coupling
		\item ASP liveness
		\item Reification
	\end{itemize}

	\section{Tool Integration}
		Allow for artists and designers to annotate in-game assets --- provide meta-data detailing structural information such as hollow hulls or potentially weak connections. As a possible extension it may be feasible to analytically determine these features from mesh information alone, without requiring user interaction, or as part of a mixed-initiative interface that allows developers to curate aesthetically suitable solutions \cite{yannakakis2014mixed}.

		Require integration with existing Xed pipeline for custom Xii engine - fortunately this makes importing metadata into the game trivial.

		Loose coupling \cite{lee2013decoupling}
	\section{Solver System}

	\section{Damage Rendering}

	\section{The Problem of Scale}
		Show off how much better this approach is ^^

\chapter{Evaluation}
	% How to evaluate your approach. Many HCI proposals will involve experimentation, which should be mentioned here. For software projects performance may be an issue and/or testing of functionality. How you will compare your approach to other methods can be described in this section.
	Paragraph on the importance of evaluation

	\section{Expressive Range and Fidelity}
		\cite{smith2010analyzing} analysing the expressive range of a level generator

	\section{Performance}
		integration with a real-time game means that the timely performance of the system is as serious concern. It may become appropriate to perform compile time optimisations, pre-calculation and pre-caching of possible solutions in order to ensure that during a match the system is able to provide appropriate damage solutions in real time. 
		Writing in 2008, Boenn et. al. \cite{boenn2008automatic} suggest that while some of the faster ASP solvers at the time were responsive enough to provide an interactive experience while generating melodies, they did not at the time provide real-time performance. There appears to be no more contemporary research that indicates significant recent speed increases, and so this is likely to remain an active research area for the project.

		Evaluation of the performance of the system within the runtime context of the game environment is most likely to be performed via frame-rate comparisons throughout development. As the project should maintain a loose coupling with the game itself, it should remain relatively easy to selectively disable it for the purpose of comparison tests across similar in-game scenarios. A range of game events may be investigated both with and without the procedural destruction system enabled, and then logs of in-game metrics including frame-rate may be analysing in order to determine the impact of the processing load required by the implemented solution.

	\section{User Acceptance Testing - Developers}

	\section{User Acceptance Testing - Players}
		As the game that the system will initially be developed within is being released on an Early Access platform, it will be possible to perform A/B testing and receive user feedback on varied incarnations of the system during development. This may range from informal feedback via community channels to focused user surveys designed to elicit specific impressions of all aspects of the game --- including the procedural destruction system. In particular, it would be useful to evaluate the impact of the system on user enjoyment and effectiveness, specifically possible increased satisfaction when applying damage to `enemy' ships and structures, and increased passive awareness of the status of their own and other visible ships for tactical decision purposes.

\chapter{Timeline}
	% Provide a proposed timeline for the project (best and worst case scenarios).
	Due to the commercial nature of the project is is likely that a completed version of the system will be needed within the next year and a half
	explain the early access system
	incremental development
	start with minimal working system within six months
	develop further guided by feedback
	after system completion, further support / development may be necessary

\chapter{Conclusion}
	% What you will have discovered by the end of the project.
	is ASP viable in real-time for solving the combinatorial explosion problem
	can we render asp solutions in a convincing manner
	is the system design sufficiently environment-agnostic


\bibliographystyle{eg-alpha-doi}

\bibliography{baththesis}


\end{document}

% 1. Specification
% The coursework consists of an approx. 8000 word research project proposal, including a literature review. The proposal must contain a description of the project to be carried out, including:
% • The project's primary and secondary goals
% • The approach to be used, including an approximate timeline
% • An initial proposal for a suitable methodology for assessing whether each of the project's goals has been accomplished
% • The motivation for the project
% • The background of the project, including any related work, whether in the form of research or applications
% • • A comprehensive literature review which supports all of the above The exact structure of the document depends on the concrete type of project, but it must include a 3-5 page Introduction section which:
% • Summarises the expected main argument of the dissertation,
% • Presents a clear roadmap for the rest of the document, and
% * Provides a proposed timeline for the project (including best & worse case scenarios)
% 2. Assessment
% The coursework will be conducted individually. Attention is drawn to the University rules on plagiarism in the Student Handbook.
% The coursework is worth 90% of the marks of the course. The remaining 10% is for active participation in the course and presentation.
% The average amount of time the student is intended to spend on the coursework is 120 hours.
% The assessment will be carried out by the Unit Lecturer together with the supervisor of the respective research project.
% Key issues for marking will be:
% • Is there a clear thesis argument for the proposal?
% • Is there a clear proposal for research that will contribute to the argument?
% • Has the author taken into account the possibility of multiple outcomes?
% • Does the Introduction give a good guide to the rest of the document?
% • Is the Bibliography properly referenced and formatted?
% • Does the project described in the proposal reflect the discussions you have had with its author?
% • Is the literature review appropriate and adequate for the project? Are references current, relevant and from acceptable sources?
% • Is the length of the proposal appropriate, and is all the content relevant?